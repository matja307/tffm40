% !TEX encoding = MacOSRoman

\documentclass[a4paper,12pt]{article}

\usepackage{listings}
\usepackage[parfill]{parskip}
\usepackage{float}
\usepackage{graphicx}
\usepackage[swedish]{babel}
\usepackage[applemac]{inputenc}
\usepackage[margin=35mm]{geometry}
\usepackage{fancyhdr}
\usepackage[encapsulated]{CJK}
\usepackage{inconsolata}
\usepackage{amssymb,amsmath}


\newenvironment{ppl}{\fontfamily{ppl}\selectfont}{}
\newenvironment{inconsolata}{\texttt}{}


\DeclareGraphicsRule{.tif}{png}{.png}{`convert #1 `dirname #1`/`basename #1 .tif`.png}

\begin{document}

\begin{titlepage}
\begin{center}


\textsc{\Large Link�ping University}

\rule{\linewidth}{0.5mm} \\[2cm]

\textsc{\Huge Title}\\[10cm]

\normalsize Mattias Jansson - matja307@student.liu.se\\[0.5 mm]
Simon Larsson - simla804@student.liu.se\\[0.5 mm]
Link�pings universitet\\[0.5 mm]
Link�ping\\[0.5 mm]
2014-03-26\\

\end{center}
\end{titlepage}

\newpage

\section{Abstract}
\emph{Express subject of the results}


\section{Introduction}
\emph{Includes: General background, Literature review, gap sentence, statement of purpose}

\section{Experimental Details}
\emph{What did I do? (Must include sufficient detail so that the results can be duplicated.}

\section{Results}
\emph{What did I get? What did I observe?}

\section{Discussion}
\emph{So what? Interpret the results.}

\section{Conclusion}
\emph{Think of three things that the reader should remember.}

\end{document}








































